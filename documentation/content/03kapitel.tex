\chapter{Component Description}\label{ch:ComponentDescription}
\section{Node}\label{Node}
Each node in the system will check on startup if there are other nodes already running within the network. This is done via a broadcast message containing the nodes ID and IP Address. If after a timeout no host responded the system assumes it's the first node and becomes leader. If other systems are already active the leader responds to the broadcast with the current list of hosts, it's own IP (the leader IP). The new host updates its own host list with the information received from the leader and sets the leader IP address. 

Each node sends regularly heartbeats to all other nodes. It then checks if there are missed heartbeats from other nodes. If for a node the heartbeats are missed for to long it removes the host from the hosts list.
If a node detects a failure of the leader a new election is started.

Each node listens for business requests from clients. If a business request is received by a host not being leader it forwards the request to the leader and waits for a response. Which is then forwarded to the requesting client.

\section{Leader}\label{sec:Leader}
The first leader is determined by the first host in the system. 
In case the leader fails a new leader is elected out of all nodes using the bully algorithm to determine a uniqe leader.

The leader behaves the same as all other nodes regarding heartbeat and processing of commited messages.

In addition to the above the leader responds to broadcast messages from new hosts and clients with the current hosts list.

Processes all business logic messages and distributes them using ordered reliable multicast to the other nodes. After a message is commited and processed the leader responds to the host which sent the message.



\section{Client}\label{sec:Client}
The client is a web interface that can be used to interact with the warehouse system.
Since only a handful of functions are needed, the interface was created using Flask, a micro web framework written in Python.
Futhermore Bootstrap was used to create a simple but reliable user interface without the need of custom CSS.
In addition, the interaction between the two components is simple and straightforward to implement.

The interface offers four functions on three pages: All available items are listed, an item can be created, and items can be bought or sold. 
Triggering a function causes the client to send a request to all currently active hosts. 
An active host is selected at random.
The requested information is then sent to this host and the response is awaited.

The requests consist of a type that triggers which server routine is requested, 
followed by the corresponding metadata, which comes from the user via a form. 
The forms already have validators so that the data does not need to be checked again.