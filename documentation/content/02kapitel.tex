\chapter{Project Requirement Analysis} \label{ch:ProjectRequirementAnalysis}
\section{Architectural Description} \label{sec:ArchitecturalDescription}
Client-Server Architecture: Our system will adopt a client-server model, which consists of multiple clients interfacing with a cluster of server nodes. The clients are responsible for querying current stock levels and submitting updates for processing. The server nodes serve these stock level requests and manage inventory updates. Amongst the server nodes, a leader will be elected to coordinate updates and ensure consistency across the distributed system.

\section{Dynamic Discovery of Hosts} \label{sec:DynamicDiscoveryOfHosts}
Server-side: Upon initiation, each server node will broadcast its presence and listen for existing members of the system to construct a current view of the cluster. This dynamic discovery protocol allows the system to scale horizontally without manual configuration.

Client-side: Clients are designed to automatically detect server nodes in the system. This enables seamless interaction with the inventory system, ensuring that clients can always locate a server node to process their requests.

\section{Fault Tolerance} \label{sec:FaultTolerance}
Our system is engineered to handle different types of failures, ensuring continuous operation:
Leader Failure: If the current leader server fails, the remaining servers will initiate a leader election to select a new leader. This ensures that the system continues to process updates without significant downtime.

Server Crash: In the event of a server crash, client requests are automatically redirected to other operational server nodes. A health check mechanism and a server list update protocol ensure that clients and servers are aware of the available nodes in real-time.

Retry Logic: To handle transient failures, the system will implement a retry mechanism. This ensures that, for instance, during a purchase processing, if a write operation fails, the system will retry the operation. We will have a limit to  prevent excessive retries that could overwhelm the system, allowing it to fail and recover.

\section{Leader Election} \label{sec:Election}
Consensus on Updates: Write operations, critical for inventory synchronisation, are managed by the leader node. Also new nodes which want to join the group exchange information with the leader. We will implement the bully algorithm for the leader election. 

\section{Ordered Reliable Multicast} \label{sec:OrderedReliableMulticast}
Our system uses FIFO ordered reliable multicast to ensure inventory updates, like purchases or deletions, are processed sequentially, maintaining data consistency and preventing access to outdated item statuses.

\begin{figure}[h!]
        \includegraphics[height=10cm, width=18cm]{images/Architecture.png}
        \caption{Architecture Diagram}
        \label{fig:architecture}
\end{figure}
